\section{Conclusions}
Data collection and surveillance has been part of American society for
centuries. Throughout the years, the public, private corporations, and the
government has reacted to developments in the data landscape. Changes in the
data landscape dictate who maintains control. Control is achieved through overt
means, state issuing of resources, and also through Panoptic means, the gaze of
private corporations. Even at tail of Moore's Law, computation, storage, and
data services continue to become less expensive, more entities will be able to
engage with the data economy. As more entities join the data economy,
surveillance is compounded and we have seen this effect with IoT. The
government's concerns in the 1960's are not disparate from the concerns of
today's, privacy and control. It is the obligation of all parties to ensure data
privacy, proper storage, and protection of individuals. Without considering the
implications discussed, the information society will continue to fall victim to
mass surveillance, thus inhibit expression through a chilling of society. As
this essay has shown, the implications can be as overtly impactful as the China
Model or as subtle as the identity issue.
