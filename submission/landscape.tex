\section{Data Landscape}
The usefulness of data in the 18-20\ts{th} centuries for creditors was making
assessments of the risk and inferences on the outcomes of loans. This was the
beginning of the commodification of personal identity. Many of the data
collection practice during that time period are analogous to those we have
today. The data landscape describes how we use, collect, and deploy data, and it
has certainly changed as the usefulness of big data has expanded across
industries. Big data are high dimensional, large datasets that are becoming
increasingly common assets to many companies. These datasets can be acquired
through purchasing many smaller datasets, or through expansive data collection
services, e.g. IoT. Since the mid 20\ts{th} century, advancements in computing have
continued to push the boundaries on what's computationally possible. Those new
computational possibilities and algorithmic discoveries have made data the most
valuable commodity to companies across many industries, which can be traded like
a financial asset.

\para{Deep Learning}
Broadly, there are 3 kinds of machine learning: supervised, unsupervised, and
reinforcement learning. For supervised learning, our data has independent
variables and outcome variables and we train our model to learn a function that
predicts the desired outcomes. For unsupervised learning, we don't know the
outcomes that we want, but we have independent variables that might have
features that will illuminate those outcomes. Finally, in reinforcement
learning, our model is trained in an environment where it gets feedback on the
outcome it decides. When the outcome is good, the model is rewarded.
Within the field of artificial intelligence, there is a subfield Machine Learning called Deep Learning.
Training a deep learning model can be thought of as supervised, unsupervised, or
semi-supervised. Many of the applications we interact with are implemented with
a deep learning model. For example, many of the algorithms that decide what
appears on a user's Facebook \texttt{newsfeed} is chosen using machine learning.

% How has deep learning changed the game?
\para{The Value of Data}
The changing value of data in the mid 20\ts{th} century has much to do with the
advent of machine learning as a technology. As this essay has shown, the value
of data and efficient means of collecting it has dictated in how it informs
the decisions made for a particular business, e.g. trustworthiness of an
individual with respect to lending. The democritization of machine learning and
data has altered the abilities of a industries. Cloud technologies have changed
the way that businesses engage with data. Microsoft, Amazon, and Google all
offer machine learning suites in cloud. It used to be the case that big data
was daunting, e.g. the compression of data sets at Tappan's agency.
Now, data storage and interacting with data has become amenable,
due to the accessibility of cloud technologies.

%The newly minted role of data scientist is evidence to rise of data influence.
%The value of data has
%increased as machine learning makes big data more amenable.
%This increase in data value
%is compounded by the democritization of cloud technologies.

% A paragraph on how data drives business decisions, but not in the
% traditional sense. How we deal with data drives the business decisions we
% makes, eg data fedderating
% \para{Data Driven Decisions}

% Might not belong here?
% Need to mention that more data will make these algorithms better
% Possibly talk about the rise of imageNet and its role in computer vision
