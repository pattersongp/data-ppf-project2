\section{Implications}

As more technology companies' data practices come to light, their users and
employees alike are becoming increasingly uncomfortable with common practice.
The data economy has created a demand for granular user data, which has become
increasingly influential in how companies become profitable. The implications
of this economy has raised ethical and moral concerns.


% How categorizing data dehumanizes and reinforces bias and racism
\para{Dehumanizing}
Francis Grund wrote of the deeply personal nature of the 20\ts{th} century
American credit system and considered this its defining feature. Critics of the
modern system point out the dehumanizing nature the credit system and many
others that rely on big data. In order for predictive models to make
classifications on or inferences about statistical populations, non-numerical
data representing a person needs to be transformed. For example, whether you're marital status is
single, divorced, or married needs to be represented as a numerical value. This
is the case for all non-numerical data. At what cost are we transforming an individual's data to fit the
models that shape the lives of many?

The categorization of data began with the \mca, where data collectors used
short hand to describe features of the data. They created rating systems
similar to the mapping of academic performance to letter
grades~\cite{lauer2017creditworthy}. In the 1960-70's, credit bureaus were faced
with the challenge of identifying individuals and called for a universal
numbering system. The obvious choice at this time was the Social Security
Number. The financial industry had already began using it as an identifier for
individuals and faced no legal challenges for non-governmental use. With the
use of this number as an identifier, the crediting agencies reduced a person to
a numerical value.

% Example of sql join on ssn
\begin{lstlisting}[language=SQL, basicstyle=\sffamily]
		SELECT ssn, name, address, creditScore
				FROM master
				WHERE ssn=@target
\end{lstlisting}

% Not sure if this belongs here.
% What am I saying here?
Proponents of these new systems feel that they are realizing American democracy
and leveling the playing field, while others consider it rise in
totalitarianism. The selling point of maintaining these data sets and
automated systems is that products become cheaper and can be used more widely.
Increased availability and decreased cost means that more socio-economic
classes are able to use these products. This was indeed the argument in the
early 20\ts{th} century credit system. Nevertheless, these systems depend on
rational groups to maintain these systems and data sets. The cost of a system
like this is privacy. As the demarcation between digital and physical spaces
becomes increasingly less apparent, the gaze of the surveillance entities is
more profound, producing a reality of constant surveillance. Americans did not
realize how much they were giving up until the late 1960's during congressional
hearings~\cite{lauer2017creditworthy}.
% Interesting note on pg 183 of Lauer

\para{Control}
While many of Representative Gallagher's prophecies were concerned with privacy,
control was also an a point of contention for the 1966 congressional hearings.
When committee heard testimony that a centralized database of American citizens would
be realized with or without their input, they were astonished. The governmental
authorities had fallen behind in control of information and the credit agencies were
pioneering the field. The dystopian future that Gallagher describes where the computerized
man is stripped of identity seemed Orwellian and far fetched in the 1960's, but
now have been realized by the Chinese Communist Party (CCP) under Xi Jinping.

In the surveillance economy, groups compete for information in order to
influence others. Under Xi Jingping, the People's Republic of China (PRC) is
dominanted by the government's information controling agaencies. The mechanisms
for control are technological infrastructure and data collection through
surveillance. The technological infrastructure is realized in 2 ways.
First, the government deploys high quality video surveillance equipment
across urban and rural areas of China. On the nose, the project was called Skynet.
This equipment is used to collect image, video, and audio data on the PRC's citizens. Second, the
CCP requires internet infrastructure organizations to enfore cencorship
standards across their physical networks. This would be the US
equivelent of Verizon dropping or blocking all requests for
\texttt{www.amazon.com}. Additionally, internet infrastructure
providers are often required to maintain their datastores within
Chinese borders, maintaining a geographical restraint on the movement
of data~\cite{qiang2019road}. The most profound digital realization of the
CCP's surveillance campaign came through the Social Credit System.

Beginning in 2014, the CCP's Social Credit System (SCS) was deployed and began
tracking the activities and subsequently offering inference on Chinese
citizens. In a regression back to the deeply personal financial credit system
of the 18\ts{th} century America, the SCS takes into account not only financial
data, but also personal behavior. The financial data is acquired through the
traditional means (e.g. cooperating credit bureaus) and personal behavior data
is collected and enhanced by infrastructural projects like Skynet. Private
corporations are complicit in the enhancement of the datasets used by the SCS
as well. Alibaba provides social networking data to illuminate networks of
individuals. The aforementioned internet providers also provide information
detailing the  internet activites of citizens, for example posting
anti-government propoganda on a social network site~\cite{qiang2019road}.
Additionally, US based companies are assisting the CCP with control through information.
In 2014, Lu Wei (CPP Internet Czar) was a guest of Facebook CEO Mark Zuckerberg and
it was later revealed that Facebook was implementing a censored version of the application,
where they could guarentee certain content was not seen on the news feeds of groups of Chinese citizens.
Then in 2017, Apple's chinese version of the \texttt{appstore} removed more than 600 VPN providers,
which would presumably be used to subvert chinese censorship (e.g. The Great Firewall of China).
Finally in 2018, details of Google's chinese search engine \texttt{Project Dragonfly} was leaked, which
search results relating to free speech, protests, and human rights~\cite{qiang2019road}.

The expanse of influence through the SCS has already effected over 15 million
people~\cite{qiang2019road}. By deploying the SCS, the CPC is able to quietly
influence large swathes of people and groups. While this is one of the overt
examples of surveillance economics enforcing totalitarianism, it is not far
from the bleek prophecies heard during the 1966 congressional hearings on a
universal citizen database.

%\para{Companies Respond}
%The common response by technology companies is the "trust us" model. Under this
%model, technology companies invoke self-governance as the response to ever
%growing concern that user data is being unethically and immorally exploited.
%This particular model may disrupt and obstruct governmental regulation. Ethical
%codes of conduct will deflect concerns, ultimately avoiding a reckoning~\cite{whittaker2018ai}
%
%Technology companies have responded in various ways. Google's CEO Sundar Pichai
%published \textit{AI at Google: our principles} in June of 2018. There, he
%outlines how Google's AI products will not reinforce bias or cause harm, and
%will be accountable, privacy sensitive, and benefit society. What is not
%outlined is what body will oversee the implementation of these principles in
%practice. Google has already struggled with composing a body that would oversee
%this, as their newly formed Ethics Board was absolved after 8 days of existence.
%
%\para{Engineers Respond}
%
