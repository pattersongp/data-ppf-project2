\section{Implications}
As more technology companies' data practices come to light, their users and
employees alike are becoming increasingly uncomfortable with common practice.
The data economy has created a demand for granular user data, which has become
increasingly influential in how companies become profitable. The implications
of this economy has raised ethical and moral concerns.

% How categorizing data dehumanizes and reinforces bias and racism
\para{Dehumanizing}
Francis Grund wrote of the deeply personal nature of the 19\ts{th} century
American credit system and considered this its defining feature. Critics of the
modern system point out the dehumanizing nature the credit system and many
other systems that rely on big data. In order for predictive models to make
classifications of or inferences about statistical populations, non-numerical
data representing a person needs to be transformed. For example, whether you're
marital status is single, divorced, or married needs to be represented as a
numerical value. This is the case for all non-numerical data. Most machine learning
packages ship with the capabilities to preprocess data in this way, as seen in figure \ref{fig:scikit}. In At what cost are
we transforming an individual's data to fit the models that shape the lives of
many? % mention scikit learn here and how packages ship with this usually?

\begin{figure}[h]
\begin{lstlisting}[language=python, basicstyle=\sffamily]
        from sklearn.preprocessing import LabelEncoder
        x_train = LabelEncoder().fit_transform(training_data)
\end{lstlisting}
\caption{An example of \texttt{scikit-learn}'s categorizing feature~\cite{scikit-learn}}
		\label{fig:scikit}
\end{figure}

The categorization of data began with \mca where data collectors used short
hand to describe features of the data. They created rating systems similar to
the mapping of academic performance to letter
grades~\cite{lauer2017creditworthy}. In the 1960-70's, credit bureaus were
faced with the challenge of identifying individuals and called for a universal
numbering system. The obvious choice at this time was the Social Security
Number. The financial industry had already began using it as an identifier for
individuals and faced no legal challenges for non-governmental use. With the
use of this number as an identifier, the crediting agencies reduced a person to
a numerical value.

% Example of sql join on ssn
\begin{lstlisting}[language=SQL, basicstyle=\sffamily]
		SELECT ssn, name, address, creditScore
				FROM master
				WHERE ssn=@target
\end{lstlisting}

Proponents of these new systems feel that they are realizing American democracy
and leveling the playing field, while others consider it a rise in
totalitarianism. The selling point of maintaining these data sets and
automated systems is that products become cheaper and can be used more broadly.
Increased availability and decreased cost means that more socio-economic
classes are able to use these products. This was indeed the argument in the
early 20\ts{th} century credit system. Nevertheless, these systems depend on
rational groups to maintain these systems and data sets. The cost of a system
like this is privacy. As the demarcation between digital and physical spaces
becomes increasingly less apparent, the gaze of the surveillance entities is
more profound, producing a reality of constant surveillance. Americans did not
realize how much they were giving up until the late 1960's during congressional
hearings~\cite{lauer2017creditworthy}.
% Interesting note on pg 183 of Lauer

\para{Identity}
One the fundamental tasks of a state is to issue identity to new citizens.
The US government began issuing social security numbers in order to identify
individual citizens.
A person cannot participate in politics or economics of the society, this has been the case for centuries.
One of the implications of the modern
surveillance state is that private companies engaging in the surveillance of
citizens provides identity to individuals with respect to technology, e.g. The
Internet. This is a fundamental flaw in our identification system, as it
requires participation in and implicit submission to surveillance.

The national registry is enhanced by the data collected through surveillance
and is not necessarily maintained by the state. As reported by the
Economist~\cite{identity2018economist}, 9/10
non-Chinese websites on the Internet either use Google, Facebook, or both as means of
identifying individuals using their services. Participation on the Internet has
become a modern requirement, just as having an identity through the state was
before the digital revolution. The Indian state has successfully implemented
a governmental system of their own: Aadhaar.

Aadhaar is a unique identifier that joins name, gender, address, date of birth,
and biometric information. Proponents of Aadhaar say that it universalizes
access to state resources. The implication of a system like this is control.
With Aadhaar deployed across India, the state has more precise control over the
population, by potentially restricting resources to communities that may have
had welfare before.

In the US, the defacto identification systems are maintained by private
organizations, which doesn't show much more promise than Aadhaar. The
difference between these two systems is control: who has control, issues
rights, and enforces regulation. Control can be achieved overtly or
through gaze alone. The China Model is the extreme case of a system similar
to Aadhaar.

\para{Control}
While many of Representative Gallagher's prophecies were concerned with
privacy, control was also a point of contention for the 1966 congressional
hearings. When committee heard testimony that a centralized database of
American citizens would be realized with or without their input, they were
astonished. The governmental authorities had fallen behind in control of
information and the credit agencies were pioneering the field. The dystopian
future that Gallagher describes where the computerized man is stripped of
identity seemed Orwellian and far fetched in the 1960's, but now have been
realized by the Chinese Communist Party (CCP) under Xi Jinping.

In the surveillance economy, groups compete for information in order to
influence others. Under Xi Jingping, the People's Republic of China (PRC) is
dominated by the government's information controlling agencies. The mechanisms
for control are technological infrastructure and data collection through
surveillance. The technological infrastructure is realized in 2 ways. First,
the government deploys high quality video surveillance equipment across urban
and rural areas of China. On the nose, the project is called Skynet. This
equipment is used to collect image, video, and audio data on the PRC's
citizens. Second, the CCP requires internet infrastructure organizations to
enforce CCP censorship standards across their physical networks. This would be the
US equivalent of Verizon dropping or blocking all requests for
\texttt{www.amazon.com}, per US policy. Additionally, internet infrastructure providers are
often required to maintain their datastores within Chinese borders, maintaining
a geographical restraint on the movement of data~\cite{qiang2019road}. The most
profound digital realization of the CCP's surveillance campaign came through
the Social Credit System.

Beginning in 2014, the CCP's Social Credit System (SCS) was deployed and began
tracking the activities of and subsequently offering inference on Chinese
citizens. In a regression back to the deeply personal financial credit system
of the 18\ts{th} century America, the SCS takes into account not only financial
data, but also personal behavior. The financial data is acquired through the
traditional means (e.g. cooperating credit bureaus) and personal behavior data
is collected and enhanced by infrastructural projects like Skynet. Private
corporations are complicit in the enhancement of the datasets used by the SCS
as well. Alibaba provides social networking data to illuminate networks of
individuals. The aforementioned internet providers also provide information
detailing the internet activities of citizens, for example posting
anti-government propaganda on a social network site~\cite{qiang2019road}.
Additionally, US based companies are assisting the CCP with control through
information. In 2014, Lu Wei (CCP Internet Czar) was a guest of Facebook CEO
Mark Zuckerberg and it was later revealed that Facebook was implementing a
censored version of the application, where they could guarantee certain content
was not seen on the news feeds of groups of Chinese citizens. Then in 2017,
Apple's Chinese version of the \texttt{appstore} removed more than 600 VPN
providers, which would presumably be used to subvert Chinese censorship (e.g.
The Great Firewall of China). Finally in 2018, details of Google's Chinese
search engine \texttt{Project Dragonfly} was leaked, which search results
relating to free speech, protests, and human rights~\cite{qiang2019road}.

The expanse of influence through the SCS has already effected over 15 million
people~\cite{qiang2019road}. By deploying the SCS, the CCP is able to quietly
influence large swathes of people and groups. While this is one of the overt
examples of surveillance economics enforcing totalitarianism, it is not far
from the bleak prophecies heard during the 1966 congressional hearings on a
universal citizen database.

\para{Chill}


%\para{Companies Respond}
%The common response by technology companies is the "trust us" model. Under this
%model, technology companies invoke self-governance as the response to ever
%growing concern that user data is being unethically and immorally exploited.
%This particular model may disrupt and obstruct governmental regulation. Ethical
%codes of conduct will deflect concerns, ultimately avoiding a reckoning~\cite{whittaker2018ai}
%
%Technology companies have responded in various ways. Google's CEO Sundar Pichai
%published \textit{AI at Google: our principles} in June of 2018. There, he
%outlines how Google's AI products will not reinforce bias or cause harm, and
%will be accountable, privacy sensitive, and benefit society. What is not
%outlined is what body will oversee the implementation of these principles in
%practice. Google has already struggled with composing a body that would oversee
%this, as their newly formed Ethics Board was absolved after 8 days of existence.
%
%\para{Engineers Respond}
%
