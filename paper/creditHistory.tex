\section{Early Data Brokers}

\para{The Mercantile Agency}
One of the earliest applications of data was credit. As the United States
expanded across North America, so did American business venture. It became increasingly more
difficult to gauge the trustworthiness of an individual you might have to do
business with, as merchants were traveling great lengths to do business, e.g. leaving your known community. In the
early 19\ts{th} century, Francis J. Grund immigrated to the United States.
Later on, he would publish \textit{The Americans} (1837) where he describes how
credit was easily rewarded in American society based on the borrower's
character and hard work. This is in opposition to the European system of
credit, where credit was awarded to those who had
collateral~\cite{lauer2017creditworthy}, e.g. houses, property, or businesses.

Around the time that Grund was publishing his de Toquevillian analysis of
American credit, there was much debate over the equality of the system.
%many parallels here to today
Many felt that the current credit system based on character, reputation, and
integrity unfairly favored the wealthy, illuminated by the 1819 and 1837 financial
panics~\cite{lauer2017creditworthy}. It was in the wake of these economic
downturns that the credit system went from a personal and subjective evaluation
system, to a formalized merit based system.

In the beginning, letters of recommendation and consensus among neighbors of
those borrowing were collected by lenders. By the 1840's, an American
merchant that did business nationally was finding that this practice was
unsustainable, due to the lack of data available. This was acutely apparent during the financial crises of the
1830's~\cite{lauer2017creditworthy}, as it brought on lending to
untrustworthy parties, resulting in defaults. Recently bankrupted, Lewis Tappan left the
silk industry, then in 1841 founded the Mercantile Agency. His goal was the
compile detailed information about businesses and potential business partners across industries.
The Mercantile Agency would travel the United States collecting personal accounts
of businessmen. Not only would they note their financial assets, but also
collect and assess accounts of character. An example of such an entry is below.

\begin{center} \textit{
		``William Goddard [of Boston] -- Save and handsome property. \$60,000 upwards.
		Very particular-- energetic in business--has influence-- apt to like strongly
		and dislike strongly"}
\end{center}

Tappan's \textit{Mercantile Agency} would serve as the model for commercial
credit reporting and is a turning point in the American surveillance
industry~\cite{lauer2017creditworthy}. The American surveillance industry,
even in its earliest form, would compile, analyze, and collect social data and
transform it into a useful piece of data that could be deployed for inference.

Early forms of data collection by creditors were what we would refer to as
spying: lurking in a tavern and taking notes of who is indulging instead of
working. This is not entirely different from ``informal data collection" by the
modern life insurance company, marking your profile as risky when you post pictures
of extreme sports on a social media account~\cite{naic2012}.

Forecasting the modern data center, Tappan's \textit{Mercantile Agency}
maintained a library in its New York office. There, subscribers and customers
could request information about most known businessmen and businesswomen.
Additionally, in order to compress the data, Tappan's clerks created a form of
shorthand, creating abstract forms of data~\cite{lauer2017creditworthy}, which
is analogous to many modern storage techniques.

% What is missing here is a look at how these companies are selling data to eachother.
%	+ Maybe talkabout the race to have the first database?
%	+ How does the subscription service work? What would be motivating here?

\para{Emerging Commodity}
In the 19\ts{th} century, the beginning of the commodification of the % XXX Should use this term earlier on.
individual is already apparent. During this time, American creditors maintained
two types of lists: the blacklist and the affirmative-negative system. The
blacklist was binary in nature, if a borrower was delinquent on payments then
they would appear on the blacklist. A type of blacklist was maintained by both shop
owners and creditors, which they shared amongst each other. The second, the
affirmative-negative system, was more granular and reflects the modern
equivalent of a credit report. This collection of information was a record of
both successful and unsuccessful borrowing and provided a complete and granular
report of an individual. Lewis Tappan of the \mca developed this style of
reporting~\cite{lauer2017creditworthy}. The advantage of this style of
reporting was that the creditors could identify groups by geography,
socio-economic class, race, or other common features, then make a determination
on whether or not to focus their efforts on that group or region.

The implications of affirmative-negative system are profound and forecasts
the systems we have today. This type of reporting is truly the
beginning of the commodification of the individual, and is an important
waypoint in the history of data surveillance. As opposed to the discrete,
binary system of trustworthiness created by the blacklist, the
affirmative-negative system reports the individual granularly, mapping them on
a continuous range of trustworthiness, for how much debt, and for how long. It also
allows for the possibility of making assumptions about individuals based
on the population or community they are a part of.

\para{Congressional Reaction}
In 1966, a congressional committee was convened to assess the implications of a
centralized federal database. The hearings were chaired by New Jersey
Representative Cornelius Gallagher and motivated by the results of a study
completed the Yale economist Richard Ruggles. In the so called Ruggle
Report~\cite{ruggles1965report}, a database to be shared among federal agencies
was recommended to be housed within a federal datacenter. The concept of a
centralized database containing complete information for every American citizen
was terrifying to the committee. Their primary concern was privacy: ``The
Computerized Man would be stripped of all his individuality and privacy"
Representative Gallagher proclaimed~\cite{ruggles1965report}. The committee did
not realize that this database already existed and was in production use at
many private corporations already. This came to light as a representative of
the Rand Corporation pointed out that nongovernmental groups were already
compiling such a database, with or without the government's
oversight~\cite{congress1967privacy}.
% analogy to today's hearings

In the years that followed, the committee would call executives of the primary
credit reporting agencies to testify about their information handling
practices. It became clear that not one single agencies knew everything about
all Americans, but with little effort, one could request information from the 3
major credit reporting agencies, join this information, and compile a complete
profile. This brought on more prophesying from Representative Gallagher, ``we
are really reaching is a total surveillance society and a totally managed
society where all information goes into a centrally located
database"~\cite{congress1967privacy}.

% XXX feel like I could say more here about congressional reaction.

