\section{Early Data Brokers}

%https://books.google.com/books?hl=en&lr=&id=UDEtDwAAQBAJ&oi=fnd&pg=PT7&dq=Creditworthy:+A+History+of+Consumer+Surveillance+and+Financial+Identity+in+America&ots=qUvEl-tBBm&sig=AmPQLBXNhBIo1_RmUmhucWR1N3w#v=onepage&q&f=false
\para{Credit Scores}
One of the earliest applications of data was credit. In the early 19\ts{th}
century, Francis J. Grund immigrated to the United States. Later on, he would
publish \textit{The Americans} (1837) where he describes how credit was easily
rewarded in American society based on the borrower's character and hard work.
This is in opposition to the European system of credit, where credit was
awarded to those who had collateral~\cite{lauer2017creditworthy}.

Around the time that Grund was publishing his de Toquevillian analysis of
American credit, there was much debate over the equality of the system.
%many parallels here to today
Many felt that the current credit system based on character, reputation, and
ethic unfairly favored the wealthy, seen in the 1819 and 1837 financial
panics~\cite{lauer2017creditworthy}.  It was in the wake of these economic
downturns that the credit system went from a personal and subjective evaluation
system, to a formalized economic system.

In the beginning, letters of recommendation were collected by credit agencies.
