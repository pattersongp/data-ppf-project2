\section{Implications}

As more technology companies' data practices come to light, their users and
employees alike are becoming increasingly uncomfortable with common practice.
The data economy has created a demand for granular user data, which has
increasingly influential in how companies become profitable. The implications
of this economy has raised ethical and moral concerns.

% How categorizing data dehumanizes and reenforces bias and racism
\para{Dehumanizing}

\para{Companies Respond}
The common resposne by technology companies is the "trust us" model. Under this
model, technology companies invoke self-governance as the response to ever
growing concern that user data is being unethically and immorally exploited.
This particular model may disrupt and obstruct governmental regulation. Ethical
codes of conduct will deflect concerns, ultimately avoiding a reckoning~\cite{whittaker2018ai}

Technology companies have reponded in various ways. Google's CEO Sundar Pichai
published \textit{AI at Google: our principles} in June of 2018. There, he
outlines how Google's AI products will not reinforce bias or cause harm, and
will be accountable, privacy sensitive, and benefit society. What is not
outlined is what body will oversee the implementation of these principles in
practice. Google has already struggled with composing a body that would oversee
this, as their newly formed Ethics Board was absolved after 8 days of existence.

\para{Engineers Respond}

