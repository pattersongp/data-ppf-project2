\begin{center}
\textit{
The Computerized Man would be stripped of all his individuality and privacy...His status\\
in society would be measured by the computer, and he would lose his personal identity.} \\
The Computer and Invsasion of Privacy, Congressional Hearing, July 1966
\end{center}

\section{Introduction}

Data collection and surveillance has been part of American society for
centuries. Starting in the early 19\ts{th} century, a need for evaluating a
businessman or businesswomen's trustworthiness carved out a new industry,
credit reporting and thus the need for surveillance. In the middle of the 20\ts{th} century, Americans realized
that massive credit reporting organizations were collecting data on their
daily lives, qualitatively evaluating their trustworthiness, then using that
data to shape their future. This was as much a revelation for politicians as it
was the general public~\cite{lauer2017creditworthy}. The computational revolution of the 1950's exaggerated
this phenomenon. With the continued advancements in computing, it became
increasingly easier to not only collect data, but also store and make
inferences from data. It was during this time that statistical credit scoring
was established and a statistical model would decide whether or not you would
be able to receive a loan for your new house.

In the 21\ts{st} century, the American public and politicians are experiencing
another reckoning. As engineers have continued to fulfill Moores' Law,
computation has become increasingly less expensive. This, just like the digital
revolution of the mid 20\ts{th} century, has given rise to the systematic
surveillance of the American public by more companies across many different industries. The growth of Internet-of-Things (IoT)
technologies compounds this phenomenon.

Contrary to the conventional understanding of the revelatory aftermath of the
2016 Presidential Election and technology companies' influence, the American
public has always known that they're being surveilled on the Internet. The
truth is that they didn't care. Participation in modern society requires your
submission data collection. This is apparent by the need to authenticate a user's
identity to use an application. Over 90\% of non-Chinese websites offer
authentication through Google or Facebook~\cite{identity2018economist}. To
prove your identity is often to share your personal information stored with
Google or Facebook through an application you want to use. Those companies
surveilling us, just like the credit bureaus, are shaping our daily lives,
our democracy, and our culture. As this essay will show, an analysis of the
practices of early credit bureaus illuminates our understanding of data
collection practices today. Furthermore, the congressional and public reaction
to the 1960's congressional hearings on credit agencies helps frame our
understanding of technology's role in our lives today.

First, a historical overview of the credit reporting and the Congressional
hearings during the 1960's digital revolution will provide a framework for
which to understand the hearings of today. Then by showing the relationship
that our data has to today's most influential technologies, we'll see that the
reactions to the 1960's hearings inform how we might respond to our current
issues with data.

Furthermore, the surveillance economy has produced a state where groups are
able to influence and shape the lives of individuals through data aquisition
and surveillance. The aquisition of data implies both surveillance and participation in data
technologies which are used to store, process, and deploy our data. The inherent
surveillance in our modern society creates a Foucodian state where a company's gaze is cowing
populations and creating a chilling effect~\cite{richards2012dangers} of speech and activity.

Finally, the world has become increasingly reliable upon data. The applications that
shape our modern experience rely on our data-- lots of data. Our daily
interactions are dictated by data which describes us. Hardware was the most
valuable item in industry, which was eventually replaced by software. Now the
landscape has changed again: the next generation's most valuable industry asset
is and will continue to be data~\cite{janeway2018doing}.

% XXX
%Succinctly, the commidification of the personal identity started with credit
%bureaues and the techniques and events associated with them inform our
%understanding of modern data collection.

% Target topic? XXX
%%%%%%%%%%%%%%%%%%%%%%%%%%%%%%%%%%%%%%%%%%%%%%%%%%%%%%%%%%%%
%It is the burden of this essay to
%show that the revelations of the 1960's and the legislation that would follow % EOMC or something?
%informs the conversation surrounding our political,
%societal, and personal response to the Data State. % I'm making this phrase up.

% XXX How clueless are we?
% It would be great to mention the congressman asking about the iPhone to Zuckerberg.

%The data economy has changed the companies do business. The profits of
%companies dealing data-- whether it be its collection or distribution--are sky
%rocketing. Alphabet (Google), Amazon, Facebook, Apple, and Microsoft net over
%\$25bn in net profit only in the first quarter of 2017~\cite{economist2017}.
%While companies of this size have been targeted by anti-trust hawks in the past
%(eg. Standard Oil), the commodity has changed from oil to data.

%The world has become increasingly reliable upon data. The applications that
%shape our modern experience rely on data-- lots of data. Our daily interactions
%are dictated by data which describes us. Hardware was the most valuable item in
%industry, which was eventually replaced by software. Now the landscape has
%changed again: the next generation 's most valuable industry asset will be
%data~\cite{janeway2018doing}.

%As the data race continues, many problematic questions are left to be
%addressed. What do we do about non-consensual data collection? When a
%technology becomes so engrained in our social fabric, what is our relationship
%with the company that owns that technology?

%Employees at large technology companies are beginning to disrupt the disrupters.
%There is an ever growing concensus among engineers, product managers, and technologists
%that large technology companies need to be regulated.

%It is the burden of this essay to show that the data economy is driven by the need
%to categorize individuals. This need is motivated by many different industries,
%advertising and credit reporting for example. Through the lense of early credit reporting,
%this essay will show how data is used to fulfill this need and how this economy
%was created and is sustained.

%The professionals that were collecting this data were astonished at the complete
%ignorance of the public that this was going on.
%~\cite{ 212 of CreditWorthy }
