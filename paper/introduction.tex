\section{Introduction}

The new data economy has changed the companies do business. The profits of
companies dealing data-- whether it be its collection or distribution--are sky
rocketing.

Alphabet (Google), Amazon, Facebook, Apple, and Microsoft net over \$25bn in
net profit only in the first quarter of 2017~\cite{economist2017}.

While companies of this size have been targeted by anti-trust hawks in the past
(eg. Standard Oil), the commodity has changed from oil to data.

The world has become increasingly reliable upon data. The applications that
shape our modern experience rely on data, lots of data.

Hardware was the valuable item in industry, which was eventually followed by
software. Now the landscape has changed again: the next generations most
valuable industry asset will be data~\cite{janeway2018doing}.

As the data race continues, many problematic questions are left to be
addressed.  What do we do about non-consensual data collection? When a
technology becomes so engrained in our social fabric, what is our relationship
with the company that owns that technology?

Employees at large technology companies are beginning to disrupt the disrupters.
There is an ever growing concensus among engineers, product managers, and technologists
that large technology companies need to be regulated.

It is the burden of this essay to show that...
